In this paper a stochastic method for computing $\pi$ is introduced and analyzed.
The Idea is that the area of the unit circle is given by $\pi\approx 3.1416$ and the area of a square
with a width of $2$ is $4$. Using this it is possible to calculate $\pi$ by generating
$P$ two-dimensional points (from a uniform distribution) inside the square and counting how many of them also
are located inside the unit circle.\par
The points are labeled $\vec r_p = (x_p, y_p)$ with $\vec r_p \in [-1, 1]\times [-1 ,1]$.
By using the Iverson Bracket
\[
	\iv{X} = \begin{cases}
		1 & \text{if X is true} \\
		0 & \mathrm{otherwise}
	\end{cases}
\]
We can write the formula for $\pi_x$, where the $x$ stands for the experiment, as
\begin{equation}
	\pi_x = \text{E}[4\iv{X^2 + Y^2 \leq 1}] = \frac{4}{P}\sum_{p = 1}^P \iv{x_p^2 + y_p^2 \leq 1}.
	\label{eq:pi}
\end{equation}
The experiment can be repeated $X$ times, so that we get a final answer
\begin{equation}
	\pi_\mathrm f = \frac{1}{X}\sum_{x = 1}^X \pi_x \qquad
	\Delta \pi_\mathrm f = \sqrt{\frac{1}{X-1}\sum_{x=1}^X (\pi_\mathrm f - \pi_x)^2}.
	\label{eq:pi_final}
\end{equation}