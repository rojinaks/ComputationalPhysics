\documentclass{article}
\usepackage{requirements}

\author{Gabriel Remiszewki \\ Rojin Aksu}
\date{$25^{\mathrm{th}}$ October 2024}
\title{Independent Trials and Trial Size \\ \large Computational Physics Exercise 1}

\begin{document}
    \maketitle

    In this paper a stochastic method for computing $\pi$ is introduced and analyzed.
    The Idea is that the area of the unit circle is given by $\pi$ and the area of a square 
    with a width of $2$ is $4$. Using this it is possible to calculate $\pi$ by generating 
    $P$ two-dimensional points (from a uniform distribution) inside the square and counting how many of them also 
    are located inside the unit circle.\par
    The points are labeled $\vec r_p = (x_p, y_p)$ with $\vec r_p \in [-1, 1]\times [-1 ,1]$.
    By using the Iverson Bracket 
    \[
        [P] = \begin{cases}
           1 & \text{if P is true}\\
           0 & \mathrm{otherwise} 
        \end{cases}
    \]
    We can write the formula for $\pi_x$, where the $x$ stands for the experiment, as 
    \begin{equation}
        \pi_X = \frac{4}{P}\sum_{p = 1}^P [x_p^2 + y_p^2 \leq 1].
        \label{eq:pi} 
    \end{equation}
    The experiment can be repeated $X$ times, so that we get a final answer 
    \begin{equation}
        \pi_\mathrm f = \frac{1}{X}\sum_{x = 1}^X \pi_x \qquad 
        \Delta \pi_\mathrm f = \sqrt{\frac{1}{X-1}\sum_{x=1}^X (\pi_\mathrm f - \pi_x)^2}.
        \label{eq:pi_final}
    \end{equation}
    

\end{document}